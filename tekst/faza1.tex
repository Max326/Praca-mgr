\section{Detekcja}

Zdecydowałem się na detekcję piłki do piłki nożnej bo jest to prosty obiekt który 
nie będzie wymagał dużego nakładu pracy a nauczę się jak budować dataset i uczyć sieć. 
Wybrałem yolov8 bo jest proste w użyciu i dobrze działa.

yolo z ultralytics -> pc -> 110-140fps, rpi?

onnx działało słabo -> 3-4fps

ncnn + ryzen 5 3600 -> ~22fps 

ncnn + rpi5 bez hat kita ~10-11fps

Zabrałem się za stworzenie gimbala w symulacji, żeby móc symulować śledzenie obiektów 
przez drona przy użyciu gimbala i bez konieczności jeżdżenia na testy


Pomysły na dalsze prace:
\begin{itemize}
    \item śledzenie w symulacji aruco markerów z ruchomym gimbalem
    \item odpalanie detekcji tylko co n klatek
    \item KCF
    \item KCF + Detekcja
\end{itemize}
    
\section{Kinematyka}

Pitch liczony zgodnie z kartezjańskim układem współrzędnych, gdzie oś Z jest skierowana w górę.
Czyli dodatni pitch oznacza, że dron jest pochylony do przodu.

\begin{itemize}
    \item $h$ - wysokość drona nad ziemią
    \item $\alpha_{drone}$ - kąt pitch drona
    \item $\alpha_{gimbal}$ - kąt pitch gimbala
    \item $\gamma$ - yaw drona
    \item $d_{ground}$ - odległość na ziemi od punktu pod dronem do punktu na ziemi, na który patrzy kamera
    \item $r$ - odległość od drona do punktu na ziemi, na który patrzy kamera
    \item $\epsilon$ - pole widzenia kamery w poziomie
    \item $\sigma$ - pole widzenia kamery w pionie
    \item $x$ - szerokość obrazu z kamery w pikselach
\end{itemize}

Dystans po ziemi:

\begin{equation}
    \frac{h}{d_{ground}} = \tan(\alpha_{drone} + \alpha_{gimbal})
\end{equation}

\begin{equation}
    d_{ground} = \frac{h}{\tan(\alpha_{drone} + \alpha_{gimbal})}
\end{equation}

Yaw:

% Wektor od drona do punktu na ziemi w celu skalowania:
% \begin{equation}
%     \sin(\alpha) = \frac{h}{r}
% \end{equation}

% \begin{equation}
%     r = \frac{h}{\sin(\alpha)}
% \end{equation}

Różnica w yaw drona dla odległości poziomej $dx$ obiektu od środka osi symetrii obrazu z kamery:

\begin{equation}
    \frac{dx}{x} = \frac{d \gamma}{\epsilon} % \tan(d \gamma)
\end{equation}

\begin{equation}
    d \gamma = \frac{dx \cdot \epsilon}{x}
\end{equation}

Co dla omawianego przypadku, gdzie pole widzenia kamery w poziomie wynosi $100^o$,
a szerokość obrazu z kamery w pikselach to $640$, daje:

\begin{equation}
    d \gamma = \frac{dx \cdot 100^o}{640}
\end{equation}

Trzeba jakoś zrealizować zatrzymywanie na określonej odległości od obiektu.

Dla cofania (kiedy gimbal osiąga maksymalny kąt pitch w dół):

\begin{equation}
    \frac{d}{h} = \tan (\frac{\sigma}{2})
\end{equation}

\begin{equation}
    d = h \tan (\frac{\sigma}{2})
\end{equation}

A więc

\begin{equation}
    \frac{dy_{px}}{480/2} = \frac{dy_m}{d_m}
\end{equation}

\begin{equation}
    dy_m = \frac{d_m \cdot dy_{px}}{480/2}
\end{equation}