\section{Detekcja}

Zdecydowałem się na detekcję piłki do piłki nożnej bo jest to prosty obiekt który nie będzie wymagał dużego nakładu pracy a nauczę się jak budować dataset i uczyć sieć. Wybrałem yolov8 bo jest proste w użyciu i dobrze działa.

yolo z ultralytics -> pc -> 110-140fps, rpi?

onnx działało słabo -> 3-4fps

ncnn + ryzen 5 3600 -> ~22fps 

ncnn + rpi5 bez hat kita ~10-11fps

zabrałem się za stworzenie gimbala w symulacji, żeby móc symulować śledzenie obiektów przez drona przy użyciu gimbala i bez konieczności jeżdżenia na testy