% Akronim: nazwa symboliczna (dla LaTeXa), nazwa skrócona, nazwa pełna
% \newacronym{IEEE}{IEEE}{Institute of Electrical and Electronics Engineers}
% \newacronym{CPEE}{CPEE}{Computational Problems of Electrical Engineering}
% \newacronym{PW}{PW}{Politechnika Warszawska}
% \newacronym{IETiSIP}{IETiSIP}{Instytut Elektrotechniki Teoretycznej i Systemów Informacyjno-Pomiarowych}

\newacronym{PID}{PID}{Regulator proporcjonalno-całkująco-różniczkujący}
\newacronym{BSP}{BSP}{Bezzałogowy Statek Powietrzny}
\newacronym{POI}{POI}{Point Of Interest}
\newacronym{GPS}{GPS}{Global Positioning System}
\newacronym{IMU}{IMU}{Inertial Measurement Unit}
\newacronym{PCB}{PCB}{Printed Circuit Board}
\newacronym{IR}{IR}{Infrared}
\newacronym{AHRS}{AHRS}{Attitude and Heading Reference System}
\newacronym{EiTI}{EiTI}{Wydział Elektroniki i Technik Informacyjnych}

\newacronym{GUI}{GUI}{Graphical User Interface}

\newacronym{IMGui}{IMGui}{Immediate Mode Graphical User Interface}
\newacronym{IMPlot}{IMPlot}{Immediate Mode Plotting}

\newacronym{laser}{LASER}{Light Amplification by Stimulated Emission of Radiation}

% \newglossaryentry{symb:Phi}{
% name=$\phi$,
% description={foobar.},
% sort=symbolphi, type=symbols
% }

% \newglossaryentry{test}{name={TeSt}, description={testowanie}}

% \newglossaryentry{symb:I}{
% name={\ensuremath{I}},
% description={Natężęnie prądu elektrycznego.},
% sort=symbI, type=symbols
% }

% \newglossaryentry{symb:Pi}{
% name=$\pi$,
% description={Stała matematyczna równa stosunkowi długości okręgu do jego średnicy.},
% sort=symbolpi, type=symbols
% }

% \newglossaryentry{symb:rho}{
% name=$\rho$,
% description={Opór właściwy.},
% sort=symbolrho, type=symbols
% }

% \newglossaryentry{symb:ri}{
% name = $\pmb{r}^{(i)}$,
% description = {Wektor położenia w układzie $\pi_i$},
% sort = symbolri, type = symbols
% }

% \newglossaryentry{symb:si}{
% name = $\pmb{s}^{(i)}$,
% description = {Wektor położenia, stały w danym układzie $\pi_i$},
% sort = symbolsi, type = symbols
% }

% % \newglossaryentry{symb:v}{
% % name = $v$,
% % description = {Prędkość liniowa},
% % sort = symbolv, type = symbols
% % }

% % \newglossaryentry{symb:a}{
% % name = $a$,
% % description = {Przyspieszenie liniowe},
% % sort = symbola, type = symbols
% % }

% \newglossaryentry{symb:alpha}{
% name = $\alpha$,
% description = {Kąt},
% sort = symbolalpha, type = symbols
% }

% \newglossaryentry{symb:omega}{
% name = $\omega$,
% description = {Prędkość kątowa},
% sort = symbolomega, type = symbols
% }

% \newglossaryentry{symb:epsilon}{
% name = $\epsilon$,
% description = {Przyspieszenie kątowe},
% sort = symbolepsilon, type = symbols
% }

% \newglossaryentry{symb:Ri}{
% name = $\pmb{R}^{i-1}_i$,
% description = {Macierz cosinusów (rotacji) z układu $(i-1)$ do $(i)$},
% sort = symbolRi, type = symbols
% }

% \newglossaryentry{symb:Ti}{
% name = $\pmb{T}^{i-1}_i$,
% description = {Macierz transformacji zawiera w sobie obroty i przesunięcia},
% sort = symbolTi, type = symbols
% }

% %% ogarnąć te układy żeby się zgadzały z moim modelem

% % na pewno 0xz w tej prostopadłości?
% \newglossaryentry{symb:PiNED}{
% name = $\pi_n$,
% description = {Nawigacyjny układ współrzędnych NED -- związany z Ziemią, początek
% układu $0_n$ znajduje się w dowolnie wybranym punkcie na powierzchni Ziemi, a oś $z$ tego układu ma kierunek i zwrot wektora przyspieszenia siły ciężkości w punkcie $0_n$. Płaszczyzna $0_nx_nz_n$ tego układu jest prostopadła do wektora przyspieszenia siły ciężkości w punkcie $0_n$, a oś $x$ tego układu jest skierowana w stronę północy geograficznej i zwrócona w stronę Bieguna Północnego. Oś $y$ tego układu dopełnia układ do prawoskrętnego i skierowana jest na wschód.},
% sort = symbolPiNED, type = symbols
% }

% \newglossaryentry{symb:PiGraw}{
% name = $\pi_g$,
% description = {Grawitacyjny układ współrzędnych -- związany z poruszającym się obiektem, początek układu $0_g$ przyjęty został w środku ciężkości~\gls{BSP}. Układ współrzędnych jest przesunięty względem układu $\pi_n$, a zwroty osi obu układów są zgodne.},
% sort = symbolPiGraw, type = symbols
% }

% \newglossaryentry{symb:PiBase}{
% name = $\pi_b$,
% description = {Układ współrzędnych obiektu -- w tym przypadku statku powietrznego -- związany z poruszającym się obiektem, początek układu $0_g$ przyjęty został w środku ciężkości~\gls{BSP}. Oś $0_bx_b$ tego układu leży w płaszczyźnie symetrii statku powietrznego $0_bx_bz_b$ i jest zwrócona do przodu kadłuba. Oś $0_bz_b$ znajduje się w płaszczyźnie symetrii i jest zwrócona w kierunku ziemi. Oś $0_by_b$ dopełnia układ do prawoskrętnego i jest zwrócona w prawą stronę.},
% sort = symbolPiBase, type = symbols
% }