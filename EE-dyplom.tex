% !TEX program = xelatex
% !TeX encoding = utf8
% !TeX spellcheck = pl-PL

%%%%%%%%%%%%%%%%%%%%%%%%%%%%%%%%%%%%%%%%%%%%%%%%%%%%%%%%%%%%%%%%%%%%%%%%%%%
% Wybierz rodzaj pracy dyplomowej oraz wydział
% Pick thesis type and faculty
%%%%%%%%%%%%%%%%%%%%%%%%%%%%%%%%%%%%%%%%%%%%%%%%%%%%%%%%%%%%%%%%%%%%%%%%%%%
\pdfminorversion=7
\documentclass[thesis=mgr,faculty=eiti]{EE-dyplom} 

% thesis=[inz|mgr|bsc|msc]
%  * inz - praca inżynierska
%  * mgr - praca magisterska
%  * bsc - bachelor thesis
%  * msc - master thesis

% Skróty nazw wydziałów zgodne z~domenami internetowymi
% Abbreviations of Faculties according to Internet subdomains
% faculty=[
%	arch,
%	gik,
%	ee,
%	wip
%	]

%%%%%%%%%%%%%%%%%%%%%%%%%%%%%%%%%%%%%%%%%%%%%%%%%%%%%%%%%%%%%%%%%%%%%%%%%%%
% Konfiguracja - do personalizacji
% Configuration - to be personalized
%%%%%%%%%%%%%%%%%%%%%%%%%%%%%%%%%%%%%%%%%%%%%%%%%%%%%%%%%%%%%%%%%%%%%%%%%%%
\instytut{Instytut Informatyki}
\kierunek{Automatyka i~Robotyka}
% \specjalnosc{Robotyka}
\title{Detekcja i śledzenie obiektów na podstawie obrazu z kamery RGB dla potrzeb bezzałogowych statków powietrznych}
\engtitle{RGB camera-based object detection and tracking for UAVs}
\album{313811}
\author{Maksymilian Łuc}
\promotor{prof. dr hab. inż. Przemysław Rokita}
\date{2026}
\longdate{2026-05-01}

%\grantlicense{TRUE} % [TRUE|FALSE]

%%%%%%%%%%%%%%%%%%%%%%%%%%%%%%%%%%%%%%%%%%%%%%%%%%%%%%%%%%%%%%%%%%%%%%%%%%%
% Streszczenie pracy i~abstract.
% In case of thesis in English swap the order - English version goes first.
%%%%%%%%%%%%%%%%%%%%%%%%%%%%%%%%%%%%%%%%%%%%%%%%%%%%%%%%%%%%%%%%%%%%%%%%%%%
\streszczeniepracy{
\lipsum[1-4]

} % koniec streszczenia

\slowakluczowe{Tracking, Detekcja, Bezzałogowy statek powietrzny, C++}

\thesisabstract{
\lipsum[1-4]
} % end of abstract

\thesiskeywords{Tracking, Detection, Unmanned aerial vehicle, C++}
\usepackage{tabularx}
\usepackage{booktabs}
\usepackage{float}

% for coding
\usepackage{listings}
\lstset{
    language=C++,
    basicstyle=\ttfamily\small,
    aboveskip={1.0\baselineskip},
    belowskip={1.0\baselineskip},
    columns=fixed,
    extendedchars=true,
    breaklines=true,
    tabsize=4,
    prebreak=\raisebox{0ex}[0ex][0ex]{\ensuremath{\hookleftarrow}},
    frame=lines,
    showtabs=false,
    showspaces=false,
    showstringspaces=false,
    keywordstyle=\color[rgb]{0.627,0.126,0.941},
    commentstyle=\color[rgb]{0.133,0.545,0.133},
    stringstyle=\color[rgb]{01,0,0},
    numbers=left,
    numberstyle=\small,
    stepnumber=1,
    numbersep=10pt,
    captionpos=t,
    escapeinside={\%*}{*)}
}

%%%%%%%%%%%%%%%%%%%%%%%%%%%%%%%%%%%%%%%%%%%%%%%%%%%%%%%%%%%%%%%%%%%%%%%%%%%
% Tu zaczyna się dokument
% Here is the beginning of the document
%%%%%%%%%%%%%%%%%%%%%%%%%%%%%%%%%%%%%%%%%%%%%%%%%%%%%%%%%%%%%%%%%%%%%%%%%%%
\begin{document}
    % Strony nagłówkowe
    % Headers
    \frontpages
    
    % Właściwa treść jest w~pliku tekst/main.tex
    % Real contents is in tekst/main.tex
    \chapter{Wstęp}


% \input{tekst/cel}

% %%%%%%%%%%%%%%%%%%%%%%%%%%%%%%%%%%%%%%%%%%%%%%%%
% \section{Kolory}
% Zarządzenie nr 57/2016 JM Rektora PW definiuje kolory dla poszczególnych grup wydziałów:
% \begin{itemize}
%  \item Kolor ,,miętowy'' -- \colorbox{bud}{wydziały ,,budowlane'':}{\color{bud} Architektury, Geodezji i~Kartografii, Inżynierii Lądowej, Instalacji Budowlanych, Hydrotechniki i~Inżynierii Środowiska, Transportu.}
%  \item Kolor ,,morelowy'' -- \colorbox{mech}{wydziały ,,mechaniczne'':}{\color{mech} Inżynierii Produkcji, Mechaniczny Energetyki i~Lotnictwa, Mechatroniki, Samochodów i~Maszyn Roboczych.}
%  \item Kolor ,,słoneczny'' -- \colorbox{chem}{wydziały ,,chemiczne'':}{\color{chem} Chemiczny, Inżynierii Chemicznej i~Procesowej, Inżynierii Materiałowej.}
%  \item Kolor ,,szafirowy'' -- \colorbox{elek}{wydziały ,,elektryczne'':}{\color{elek} Elektryczny, Elektroniki i~Technik Informacyjnych.}
%  \item Kolor ,,śliwkowy,'' -- \colorbox{mfiz}{wydziały ,,matematyczno-fizyczne'':}{\color{mfiz} Fizyki, Matematyki i~Nauk Informacyjnych.}
%  \item Kolor ,,wrzosowy'' -- \colorbox{multi}{wydziały ,,multidyscyplinarne'' oraz ,,społeczno-ekonomiczne'':}{\color{multi} Administracji i~Nauk Społecznych, Zarządzania, Budownictwa,~Mechaniki i~Petrochemii; Kolegium Nauk Ekonomicznych i~Społecznych.}
% \end{itemize}


% % Rysunek przy użyciu tikz
% \begin{figure}[!ht]
%     \centering
% 	\resizebox*{0.618\linewidth}{!}{%
%         \begin{tikzpicture}every node/.append style = {anchor=center}
%             \filldraw[fill=EEBlue] (-0.5,-0.25) rectangle (10.5,8.75);
%             \fill[EEBlueLight] (0,7.5) rectangle (10,8.5);
%             % duża przerwa
%             \fill[EEBlueDark] (0,5.5) rectangle (10,6.5);
%             \fill[EEOrange] (0,4.25) rectangle (10,5.25);
%             \fill[EETangerine] (0,3) rectangle (10,4);
%             \fill[EEGold] (0,2) rectangle (10,3);
%             \fill[EEAzure] (0,1) rectangle (10,2);
%             \fill[EEUltramarine] (0,0) rectangle (10,1);
%             \node at (5,7.90) {\large \textcolor{EEBlueDark}{EEBlueLight} \#e2ecf8};
%             \node at (5,6.90) {\large \textcolor{EEBlueDark}{EEBlue} \#8ba5d3};
%             \node at (5,5.90) {\large \textcolor{EEBlueLight}{EEBlueDark} \#45536a};
%             \node at (5,4.65) {\large \textcolor{EEBlueDark}{EEOrange} \#f6a307};
%             \node at (5,3.40) {\large \textcolor{EEBlueDark}{EETangerine} \#ff7700};
%             \node at (5,2.40) {\large \textcolor{EEBlueDark}{EEGold} \#ffd100};
%             \node at (5,1.40) {\large \textcolor{EEBlueLight}{EEAzure} \#0088ff};
%             \node at (5,0.40) {\large \textcolor{EEBlueLight}{EEUltramarine} \#002eff};
%         \end{tikzpicture}
% 	}
%     \caption{Próbka zdefiniowanych kolorów}
%     \label{rys:colorsample}
% \end{figure}

% Bladopomarańczowy kolor nazwany \textcolor{EEOrange}{EEOrange} jest dopełniającym dla podstawowego koloru wydziałowego EEBlue. Ton ten leży po przeciwnej stronie koła barw niż ton podstawowy a~więc tworzy z~nim silny kontrast. Może to dawać bardzo dobry lub bardzo kiepski efekt. Ponadto dla osób z~całkowitym daltonizmem te dwie barwy mogą być trudno odróżnialne. Dlatego warto rozważyć jej łączenie z~tonem jaśniejszym lub ciemniejszym barwy podstawowej.

% Dodatkowo zdefiniowana jest tetrada barw uzupełniających, która w~połączeniu z~barwą podstawową lub jej odcieniami może dać dobry efekt wizualny. Te barwy to:
% \begin{itemize}
%     \item \colorbox{EETangerine}{mandarynkowy} -- \textcolor{EETangerine}{EETangerine} (EETangerine),
%     \item \colorbox{EEGold}{złoty} --  \textcolor{EEGold}{EEGold} (EEGold),
%     \item \colorbox{EEAzure}{lazurowy} --  \textcolor{EEAzure}{EEAzure} (EEAzure),
%     \item \colorbox{EEUltramarine}{\textcolor{white}{ultramaryna}} --  \textcolor{EEUltramarine}{EEUltramarine} (EEUltramarine).
% \end{itemize}
% Barwy mandarynkowa i~złota mogą się dobrze kojarzyć z~barwami loga Wydziału Elektrycznego.

% W pracy dyplomowej użycie barw powinno być bardzo ograniczone i~dobrze przemyślane. Kolory silnie kontrastowe, szczególnie ciepłe barwy pomarańczowo-żółte w~zestawieniu z~zimną barwą podstawową Wydziału Elektrycznego należy używać nadzwyczaj ostrożnie aby nie uzyskać niepoważnego, ,,choinkowego'' efektu.

% \subsection{Kompilowanie lokalnie}
% Darmowy dostęp do Overleaf zapewnia czas każdej kompilacji nie dłuższy niż 1 minuta. W~przypadku skomplikowanej pracy czas ten może zostać przekroczony i~wówczas Overleaf nie wygeneruje pliku wynikowego. Dlatego warto rozważyć pobranie szablonu na własny komputer z~zainstalowanym \XeLaTeX{em}. Kompilacja może wówczas wyglądać tak, jak na listingu~\ref{lst:kompilacja}.

% \begin{lstlisting}[language=bash,
%     caption={Kompilacja pracy dyplomowej lokalnie},
%     label={lst:kompilacja}]
% xelatex EE-dyplom && biber EE-dyplom && makeglossaries EE-dyplom && xelatex EE-dyplom && xelatex EE-dyplom
% \end{lstlisting}

% Polecenie \textbf{xelatex} można zastąpić przez \textbf{pdflatex} ale nie jest to zalecane.


\section{Przegląd literatury i stanu wiedzy}
\label{sec:przeglad_literatury}

\textbf{Era autonomicznych systemów latających}

Bezzałogowe statki powietrzne (BSP), potocznie zwane dronami, zrewolucjonizowały wiele dziedzin życia i przemysłu, od filmografii i rozrywki, przez rolnictwo precyzyjne, inspekcje infrastruktury, aż po zastosowania w bezpieczeństwie i obronności. Ich wszechstronność wynika w dużej mierze z możliwości przenoszenia różnorodnych sensorów, w tym kamer światła widzialnego (RGB), które stały się podstawowym źródłem informacji o otoczeniu. Wraz z rosnącą dostępnością i zaawansowaniem technologicznym BSP, kluczowym kierunkiem rozwoju staje się zwiększanie ich~autonomii \cite{bsp_autonomy_review}. Zdolność do samodzielnego postrzegania, rozumienia i reagowania na otoczenie jest fundamentem tej autonomii, a jednym z jej podstawowych elementów jest efektywna detekcja i śledzenie obiektów w czasie rzeczywistym na podstawie obrazu z kamery. Niniejsza praca koncentruje się właśnie na tym wyzwaniu, eksplorując możliwości implementacji takiego systemu na platformie BSP o ograniczonych zasobach.
\\ \newline
\textbf{Wyzwania wizyjnej detekcji i śledzenia obiektów na BSP}

Implementacja systemów wizyjnych na pokładach BSP napotyka szereg specyficznych wyzwań, które odróżniają je od stacjonarnych systemów monitoringu. Przede wszystkim, platformy latające charakteryzują się \textbf{ograniczonymi zasobami obliczeniowymi, energetycznymi oraz wagowymi} \cite{bsp_constraints_review}. To narzuca konieczność stosowania algorytmów zoptymalizowanych pod kątem wydajności, często kosztem pewnej precyzji, jak również doboru lekkich komponentów sprzętowych. 

Kolejnym istotnym problemem są \textbf{dynamiczne warunki pracy}. BSP operują w zmiennym środowisku: zmieniające się oświetlenie (słońce, cienie, chmury), warunki atmosferyczne (mgła, deszcz), a także nieprzewidywalny ruch tła i samych śledzonych obiektów stanowią poważne wyzwanie dla algorytmów wizyjnych \cite{bsp_vision_challenges}. Co więcej, sam ruch BSP oraz wibracje generowane przez silniki i śmigła wprowadzają \textbf{zakłócenia w obrazie}, takie jak rozmycie ruchu (motion blur) czy drgania, które negatywnie wpływają na jakość detekcji i stabilność śledzenia \cite{bsp_vibration_impact}. Konieczność radzenia sobie z tymi problemami motywuje do poszukiwania zarówno zaawansowanych algorytmów przetwarzania obrazu, jak i rozwiązań mechanicznych stabilizujących kamerę.
\\ \newline
\textbf{Przegląd metod detekcji i śledzenia obiektów}

Literatura oferuje szeroki wachlarz metod detekcji i śledzenia obiektów, które można podzielić na dwie główne kategorie: metody klasyczne oraz oparte na głębokim uczeniu.

\textbf{Metody klasyczne}, takie jak kaskady cech Haara \cite{opencv, viola2001rapid}, detektory cech lokalnych (np. SIFT, SURF, HOG) czy metody oparte na dopasowaniu szablonów i analizie ruchu (np. Optical Flow), charakteryzują się zazwyczaj niskimi wymaganiami obliczeniowymi \cite{classic_detection_review}. Kaskady Haara działają bardzo wydajnie, umożliwiając przetwarzanie w czasie rzeczywistym. Ich głównym ograniczeniem jest jednak mniejsza elastyczność i dokładność w porównaniu do nowszych metod, szczególnie w złożonych scenach i przy detekcji różnorodnych obiektów.

\textbf{Metody oparte na głębokim uczeniu (Deep Learning)}, w szczególności konwolucyjne sieci neuronowe (CNN), zdominowały w ostatnich latach dziedzinę detekcji obiektów, oferując znacznie wyższą skuteczność \cite{dl_detection_review}. Architektury takie jak YOLO (You Only Look Once)  \cite{yolo}, SSD (Single Shot MultiBox Detector) \cite{liu2016ssd} czy Faster R-CNN \cite{ren2015faster} osiągają imponujące wyniki na standardowych benchmarkach. Rozwój lżejszych wariantów, takich jak YOLOv4-tiny czy MobileNet-SSD \cite{howard2017mobilenets}, stara się znaleźć kompromis między dokładnością a szybkością działania. Po detekcji obiektu kluczowe staje się jego \textbf{śledzenie (tracking)}. Algorytmy śledzenia, takie jak filtry Kalmana, filtry cząsteczkowe, algorytmy korelacyjne (np. MOSSE, KCF) \cite{tracking_review} czy nowsze metody oparte na DL (np. SORT, DeepSORT) \cite{bewley2016sort, wojke2017deepsort}, pozwalają na utrzymanie tożsamości obiektu między klatkami, nawet w przypadku chwilowych zaników detekcji. Wybór odpowiedniego algorytmu śledzenia również zależy od dostępnych zasobów i wymagań aplikacji.
\\ \newline
\textbf{Stabilizacja mechaniczna i rozwiązania systemowe}

Aby przeciwdziałać negatywnemu wpływowi ruchu BSP i wibracji na jakość obrazu, powszechnie stosuje się \textbf{mechaniczne systemy stabilizacji kamery, zwane gimbalami}. Rozwiązania te sięgają od prostych, jednoosiowych konstrukcji sterowanych serwomechanizmami, takich jak te oparte na projektach open-source \cite{medlin}, aż po zaawansowane, wieloosiowe gimbale wykorzystujące silniki bezszczotkowe (BLDC), zapewniające znacznie płynniejszą i dokładniejszą stabilizację \cite{gimbaldiy}. Rozwiązania komercyjne, takie jak te oferowane przez firmę DJI \cite{dji_website}, integrują zaawansowane gimbale z algorytmami wizyjnymi, oferując gotowe systemy śledzenia i autonomicznych lotów. Jednak są to często systemy zamknięte i kosztowne. Na rynku istnieją również bardziej specjalistyczne rozwiązania, np. w sektorze obronnym (firma Anduril \cite{anduril_website}), które pokazują potencjał integracji zaawansowanej wizji i mechaniki, ale są one poza zasięgiem typowych zastosowań cywilnych i akademickich.
\newpage 
% \\ \newline
\textbf{Identyfikacja luki badawczej i uzasadnienie celu pracy}

Pomimo bogatej literatury i dostępności różnorodnych algorytmów oraz komponentów, nadal istnieje luka w zakresie kompleksowych, dobrze udokumentowanych i przetestowanych rozwiązań do detekcji i śledzenia obiektów, które byłyby zoptymalizowane pod kątem specyficznych ograniczeń niskokosztowych platform BSP. Wiele badań skupia się albo na samych algorytmach, testowanych często na wydajnych komputerach PC, albo na zaawansowanych, drogich systemach komercyjnych. Brakuje natomiast prac demonstrujących praktyczną integrację i ewaluację \textit{całego} systemu – od lekkiej mechaniki, przez dobór i optymalizację algorytmów wizyjnych (balansujących dokładność i wydajność na konkretnym sprzęcie wbudowanym), po system sterowania – w kontekście realnych ograniczeń platformy BSP.

Niniejsza praca ma na celu wypełnienie tej luki poprzez zaprojektowanie, zbudowanie i przetestowanie zintegrowanego systemu detekcji i śledzenia obiektów dla BSP. Praca skupi się na analizie kompromisu między wydajnością a dokładnością wybranych algorytmów wizyjnych w kontekście realnych zasobów sprzętowych oraz na opracowaniu pętli sterowania umożliwiającej śledzenie obiektu poprzez ruch kamery. Celem jest stworzenie funkcjonalnego prototypu, który stanowiłby podstawę do dalszych badań i rozwoju tanich, autonomicznych systemów wizyjnych dla BSP, a także dostarczenie praktycznych wniosków dotyczących implementacji takich systemów.



\section{Przegląd rozwiązań mechanicznych} \label{sec:przeglad_mech}

Przegląd rozwiązań został rozpoczęty od wyszukania rozwiązań umożliwiających sterowanie kamerą umieszczoną na BSP. Rozwiązaniem, które spełniało te wymagania, było \cite{medlin}. Było to proste rozwiązanie open source, używające serwa z prostym mechanizmem do sterowania pochyleniem kamery. Poniżej przedstawione zostały gimbale \textit{Medlin Drone} w wersjach z serwem ustawionym pionowo oraz poziomo.

\begin{figure}[H]
    \centering
    \begin{minipage}{0.48\textwidth}
        \centering
        \includegraphics[width=1.0\textwidth]{rysunki/zdjecia/medlin1.png}
        \caption{Gimbal Medlin z kamerą \textit{GoPro}, z serwem zamontowanym pionowo}
    \end{minipage}\hfill
    \begin{minipage}{0.48\textwidth}
        \centering
        \includegraphics[width=1.0\textwidth]{rysunki/zdjecia/medlin2.png} 
        \caption{Gimbal Medlin z kamerą \textit{GoPro}, z serwem zamontowanym poziomo}
    \end{minipage}
\end{figure}

Na powyższej stronie znajdują się również rozwiązania z paskiem zębatym, zamiast mechanizmu opartego na popychaniu dźwigni połączonej z kamerą. Takie rozwiązanie zostało przedstawione poniżej.

\begin{figure} [H]
    \centering
    \includegraphics[width=0.5\linewidth]{rysunki/zdjecia/gimbalzeby.png}
    \caption{Gimbal Medlin z kamerą \textit{GoPro}, z paskiem zębatym}
\end{figure}

Kolejnym rozwiązaniem znalezionym w internecie było \cite{gimbaldiy}. Przedstawione zostało na poniższym zdjęciu.

\begin{figure} [H]
    \centering
    \includegraphics[width=0.5\linewidth]{rysunki/zdjecia/diygimbal.png}
    \caption{Gimbal dwuosiowy, używający silników \textit{BLDC}}
\end{figure}

To rozwiązanie było znacznie bardziej zaawansowane, wymagało bowiem użycia silników \textit{BLDC} oraz enkoderów, a także napisania sterownika do silników. 

\newpage
\section{Przegląd rozwiązań algorytmów wizyjnych} \label{sec:przeglad_wiz}

W tym podrozdziale przeanalizowane zostały gotowe algorytmy służące do analizy obrazu. 
Do obsługi kamery została wykorzystana biblioteka \texttt{opencv}.

Najpierw przeanalizowany został algorytm \texttt{yolov4}. Pliki konieczne do uruchomienia, 
czyli: \begin{itemize}
    \item \texttt{yolov4.weights}
    \item \texttt{yolov4.cfg}
    \item \texttt{coco.names}
\end{itemize}
zostały zaczerpnięte z \cite{yolo}. Poniżej przedstawione zostały zrzuty ekranu wykonane 
podczas działania programu.

\begin{figure}[H]
    \centering
    \begin{minipage}{0.48\textwidth}
        \centering
        \includegraphics[width=1.0\textwidth]{rysunki/twarz/yolov4_1.png}
        \caption{Wykryta osoba oraz szklanka}
    \end{minipage}\hfill
    \begin{minipage}{0.48\textwidth}
        \centering
        \includegraphics[width=1.0\textwidth]{rysunki/twarz/yolov4_2.png} 
        \caption{Wykryta osoba oraz telefon}
    \end{minipage}
\end{figure}

Program działał bardzo powoli, wykonywała się 1 pętla programu w ciągu kilku sekund. 
Takie działanie programu uniemożliwiało śledzenie obiektów z dostateczną dokładnością.

Następnie przeanalizowany został algorytm \texttt{yolov4-tiny}. 
Pliki konieczne do uruchomienia, a więc: 
\begin{itemize}
    \item \texttt{yolov4-tiny.weights}
    \item \texttt{yolov4-tiny.cfg}
\end{itemize}
zostały pobrane analogicznie jak wyżej. Poniżej przedstawione zostały zrzuty ekranu wykonane podczas działania programu.

\begin{figure}[H]
    \centering
    \begin{minipage}{0.48\textwidth}
        \centering
        \includegraphics[width=1.0\textwidth]{rysunki/twarz/tiny1.png}
        \caption{Wykryte dwie osoby}
    \end{minipage}\hfill
    \begin{minipage}{0.48\textwidth}
        \centering
        \includegraphics[width=1.0\textwidth]{rysunki/twarz/tiny2.png} 
        \caption{Wykryta osoba}
    \end{minipage}
\end{figure}

Na powyższych zrzutach widać, że sieć nie radzi sobie z rozpoznawaniem przedmiotów, które nie są ludźmi -- \texttt{yolov4-tiny} określa telefon komórkowy mianem \textit{person}, czyli osoby. Program działał w tym przypadku znacznie płynniej. Takie działanie programu mogłoby umożliwić śledzenie obiektów w czasie rzeczywistym.

Ostatnim algorytmem były kaskady Haara, zaczerpnięte z \cite{opencv}. Poniżej przedstawione zostały zrzuty ekranu wykonane podczas działania programu.

\begin{figure}[H]
    \centering
    \begin{minipage}{0.48\textwidth}
        \centering
        \includegraphics[width=1.0\textwidth]{rysunki/twarz/haar1.png}
        \caption{Wykryta twarz}
    \end{minipage}\hfill
    \begin{minipage}{0.48\textwidth}
        \centering
        \includegraphics[width=1.0\textwidth]{rysunki/twarz/haar2.png} 
        \caption{Twarz wykryta pod innym kątem}
    \end{minipage}
\end{figure}

Powyższy algorytm działał zdecydowanie najwydajniej z wyżej opisanych. Jego implementacja pozwoliłaby na bezproblemowe śledzenie obiektów w czasie rzeczywistym.

\newpage
\section{Przegląd rozwiązań rynkowych}

Na rynku komercyjnym działa firma DJI -- firma zajmująca się produkcją dronów zintegrowanych z gimbalami. Rozwiązania tej firmy stosowane są popularnie w branży filmowej. Mają wbudowane algorytmy m. in. \begin{itemize}
    \item śledzenia
    \item unikania przeszkód
    \item nagrywania inteligentnych ujęć
\end{itemize}

Poniżej przedstawiony został przykładowy produkt tej firmy.

\begin{figure} [H]
    \centering
    \includegraphics[width=0.5\linewidth]{rysunki/zdjecia/djiair3.jpg}
    \caption{Dron DJI Air 3}
\end{figure}

Podobne rozwiązania funkcjonują na rynku zbrojeniowym. Są one oczywiście rozwinięte o funkcjonalności pozwalające na wyrządzenie zniszczeń. Firmą zajmującą się takimi rozwiązaniami jest Anduril. Ich dron Bolt pozwala na autonomiczne śledzenie i detonację w pobliżu celu, z wybranego przez użytkownika kąta. Poniżej przedstawiony został \gls{BSP} tej firmy.

\begin{figure} [H]
    \centering
    \includegraphics[width=0.5\linewidth]{rysunki/zdjecia/andurilbolt.png}
    \caption{Dron Anduril Bolt}
\end{figure}

\textit{nie wiem czy pisać o głowicach tu}

\newpage
\section{Przebieg pracy}

W tej sekcji opisany został przebieg pracy oraz dokonane eksperymenty.

\subsection{Założenia}

Celem projektu był system możliwy do zintegrowania z bezzałogowym statkiem powietrznym. Musiał więc być:

\begin{itemize}
    \item niewielki
    \item lekki
    \item wydajny
\end{itemize}

Dodatkowym atutem była oczywiście niska cena dobranych komponentów.

\subsection{Dobór komponentów}

Postawione wymagania wymusiły wybór konkretnych komponentów. Dobrane zostały:

\begin{itemize}
    \item Raspberry Pi 5
    \item Kamera światła widzialnego z obiektywem \texttt{IR 3MP 3.6mm, 1/2.5"}
    \item Serwo \texttt{MG-996R}
    \item Płytka \texttt{STM32F411E-Disco}
\end{itemize}

\subsection{Mechanika}

Prace rozpoczęte zostały od zamodelowania uchwytu na kamerę, pozwalającego na śledzenie wykrytych obiektów. Zamodelowany został prosty gimbal jednoosiowy, którego napędem było wymienione powyżej serwo. Programem wykorzystanym do tego celu był \textit{Fusion 360}. Części modelowane były tak, aby umożliwić łatwe wytworzenie technologią 3D.

Gimbal został dostosowany do montażu do ramy drona \textit{TBS Source One}, przedstawionej poniżej.

\begin{figure} [H]
    \centering
    \includegraphics[width=0.5\linewidth]{rysunki/zdjecia/tbscut.jpg}
    \caption{Rama TBS Source One}
\end{figure}

Poniżej przedstawione zostały zdjęcia modelu 3D uchwytu na kamerę. Część ta składa się z dwóch mniejszych części, aby ułatwić wydruk 3D. Rozstaw śrub został dostosowany do rozstawu śrub znajdujących się w uchwycie na kamerę, wykonanym przez innego członka koła naukowego.

\begin{figure}[H]
    \centering
    \begin{minipage}{0.48\textwidth}
        \centering
        \includegraphics[width=1.0\textwidth]{rysunki/zdjecia/cammount1.png}
        \caption{Uchwyt na kamerę w pierwszym rzucie}
    \end{minipage}\hfill
    \begin{minipage}{0.48\textwidth}
        \centering
        \includegraphics[width=1.0\textwidth]{rysunki/zdjecia/cammount2.png} 
        \caption{Uchwyt na kamerę w drugim rzucie}
    \end{minipage}
\end{figure}

Poniżej przedstawione zostało mocowanie serwa oraz część umożliwiająca łożyskowanie uchwytu na kamerę.

\begin{figure} [H]
    \centering
    \includegraphics[width=0.5\linewidth]{rysunki/zdjecia/servomnt1.png}
    \caption{Mocowanie serwa oraz tuleja do łożyskowania uchwytu na kamerę}
\end{figure}

Poniżej przedstawione zostało łożyskowanie gimbala (mocowania kamery do uchwytu na serwo). Użyta została śruba pasowana \texttt{ISO 7379-6-M5-30} oraz dwa łożyska \texttt{686 2Z 6x13x5}.

\begin{figure} [H]
    \centering
    \includegraphics[width=0.75\linewidth]{rysunki/zdjecia/lozyskowanie.png}
    \caption{Łożyskowanie gimbala}
\end{figure}

Gimbal zostanie zamocowany śrubami, które trzymają tuleje dystansowe w ramie, do której zostanie zamocowany. Poniżej przedstawione zostało pełne złożenie gimbala.

\begin{figure}[H]
    \centering
    \begin{minipage}{0.48\textwidth}
        \centering
        \includegraphics[width=1.0\textwidth]{rysunki/zdjecia/gimbalmoj1.png}
        \caption{Złożenie gimbala z serwem w pierwszym rzucie}
    \end{minipage}\hfill
    \begin{minipage}{0.48\textwidth}
        \centering
        \includegraphics[width=1.0\textwidth]{rysunki/zdjecia/gimbalmoj2.png} 
        \caption{Złożenie gimbala z serwem w drugim rzucie}
    \end{minipage}
\end{figure}

Poniżej przedstawione zostało zdjęcie wydrukowanego i złożonego gimbala.
\begin{figure} [H]
    \centering
    \includegraphics[width=0.4\linewidth]{rysunki/zdjecia/gimbaldruk.jpg}
    \caption{Złożenie wydrukowanego gimbala}
\end{figure}

\subsection{Integracja}

% \subsubsection{Firmware}

W pierwszej kolejności wykonane zostały próby uruchomienia sterowania serwomechanizmem bezpośrednio przy użyciu komputera Raspberry Pi. Rezultaty były jednak niezadowalające: \begin{itemize}
    \item serwo nie działało płynnie -- obecne były spore drgania po zrealizowaniu zadanego kąta
    \item niemożliwe okazało się zadawanie małych uchybów -- kąty musiały się różnić o co najmniej 5-10 stopni, w przeciwnym razie serwo nie zmieniało swojej pozycji
\end{itemize} 

Podjęta została decyzja o zaniechaniu prób uruchomienia sterowania serwem na Raspberry Pi. Zamiast tego, użyty został mikrokontroler \texttt{STM32F411E-Disco}. Komputer Raspberry Pi został więc użyty jedynie do wysyłania kąta zadanego do zrealizowania przez sterownik serwa. 

Uchyb wyliczany był na podstawie odległości środka twarzy wykrytej przez algorytm kaskad Haara od środka obrazu z kamery. Uchyb ten przemnażany był następnie przez stałą $K_p$ i dodawany do aktualnej wartości kąta, a następnie wysyłany do sterownika serwa przez \texttt{UART}. Była to imitacja regulacji w pętli zamkniętej z regulatorem $P$.

Kąt do zrealizowania był następnie obsłużony w przerwaniu na mikrokontrolerze \texttt{STM32} i generowany był sygnał \texttt{PWM}. Tym razem serwo zachowywało się bez zarzutu. Umożliwiło to integrację systemu oraz zamknięcie pętli sterowania: \begin{enumerate}
    \item otrzymany z kamery obraz był przetwarzany algorytmem kaskad Haara
    \item obliczany był uchyb -- odległość pomiędzy środkiem obrazu, a środkiem obszaru wyznaczającego twarz w osi pionowej (jest to gimbal jednoosiowy)
    \item przy użyciu regulatora $P$ wyznaczany był kąt do zrealizowania przez serwo
    \item kąt był wysyłany przez \texttt{UART} do mikrokontrolera \texttt{STM32}
    \item serwo po odebraniu kąta wysyłało do serwa sygnał \texttt{PWM}
    \item serwo realizowało zadany kąt, zbliżając środek wykrytej twarzy do środka sceny kamery
\end{enumerate} 

Zasilanie do serwa zostało zrealizowane za pomocą zasilacza laboratoryjnego, co będzie musiało w przyszłości zostać zastąpione mniejszym, kompaktowym źródłem prądu.
W przyszłości kontroler \texttt{STM32F411E-Disco} powinien również zostać zastąpiony bardziej kompaktowym kontrolerem -- np \texttt{STM32F411 Blackpill}. Model \texttt{E-Disco} został bowiem użyty ze względu na prostotę użycia -- ma wbudowany \texttt{ST-Link}, umożliwiający wgrywanie programów, zasilanie oraz debug. Jest to doskonała platforma do testów oprogramowania, ponieważ zawiera sporo portów \texttt{GPIO} oraz diod, które umożliwiają sprawdzenie, czy np. komendy wykonywane są z odpowiednią częstotliwością (przykładem może być odbiór kąta przez \texttt{UART}).
% \subsubsection{Software}

% Opisać: stm32, uart, przerwania, rpi, haar, regulator P, generowanie pwm

\section{Wnioski}

\subsection{Sekcja programistyczna}

Przeprowadzona analiza algorytmów wizyjnych wykazała, że istnieją znaczące różnice w ich skuteczności i wydajności, co znacząco wpływa na ich zastosowanie w systemach pokładowych bezzałogowych statków powietrznych.

Algorytm \texttt{yolov4} zapewnił wysoką dokładność wykrywania obiektów, lecz jego wydajność była niewystarczająca do zastosowań w czasie rzeczywistym. Długi czas przetwarzania pojedynczej klatki uniemożliwia płynne śledzenie obiektów, co czyni go niepraktycznym dla obecnie dobranych komponentów -- komputera Raspberry Pi 5. Może to być podstawą do ulepszenia sprzętu -- np. dołączenia do niego modułu \textit{AI Hat} od firmy \textit{Hailo}.

Algorytm \texttt{yolov4-tiny}, będący uproszczoną wersją \texttt{yolov4}, działał znacznie szybciej, co teoretycznie pozwalałoby na śledzenie obiektów w czasie rzeczywistym. Jednak jego dokładność była niższa – w szczególności sieć miała trudności z prawidłową klasyfikacją niektórych obiektów, co mogłoby prowadzić do błędnych decyzji w autonomicznym systemie nawigacyjnym.

Najbardziej wydajnym rozwiązaniem okazały się kaskady Haara, które zapewniły bardzo szybkie przetwarzanie obrazu. Ich dokładność była jednak ograniczona do wykrywania określonych wzorców, co może stanowić problem w przypadku bardziej złożonych scenariuszy śledzenia obiektów o różnorodnych kształtach i teksturach.

Podsumowując, wybór odpowiedniego algorytmu zależy od kompromisu między dokładnością a wydajnością. Dla zastosowań w dronach fundamentalne jest zapewnienie dostatecznej szybkości przetwarzania, co sugeruje konieczność dalszej analizy metod optymalizacji modeli głębokiego uczenia lub rozważenie alternatywnych podejść, takich jak na przykład wykorzystanie sieci neuronowych zoptymalizowanych pod kątem systemów wbudowanych.

\subsection{Sekcja mechaniczna}

Zaprojektowany jednoosiowy gimbal stanowi prostą, lecz funkcjonalną konstrukcję umożliwiającą śledzenie obiektów wykrytych przez system wizyjny. Jego modułowa budowa oraz dostosowanie do technologii druku 3D pozwalają na łatwe wytworzenie i montaż, co zwiększa elastyczność projektu i umożliwia szybkie iteracje w przypadku konieczności modyfikacji.

Integracja gimbala z ramą drona \textit{TBS Source One} została zaplanowana w sposób minimalizujący konieczność ingerencji w strukturę samej ramy. Wykorzystanie istniejących punktów mocowania (śrub tulei dystansowych) pozwala na stabilne i trwałe przymocowanie uchwytu kamery bez osłabiania konstrukcji nośnej drona.

Zastosowanie łożysk \texttt{686 2Z 6x13x5} oraz śruby pasowanej \texttt{ISO 7379-6-M5-30} w mechanizmie obrotu gimbala zapewnia płynność ruchu oraz redukcję luzów, co umożliwi precyzyjne śledzenie obiektów. 

Podsumowując, opracowana konstrukcja gimbala stanowi solidną bazę do dalszej optymalizacji. Możliwe kierunki ulepszeń obejmują zastosowanie silnika BLDC z enkoderem, bądź zapewnienie systemu tłumienia drgań drona, w celu uzyskania lepszej jakości obrazu.

\section{Plan dalszych prac}

% Serwomechanizm sterowany będzie przez komputer Raspberry Pi 5, który będzie przesyłał informację o koniecznym obrocie kamery do płytki \texttt{STM32F411E-Disco}, która będzie sterować serwem, w celu zapewnienia płynnych przebiegów sygnału \texttt{PWM}.

Zaproponowana konstrukcja stanowi dobrą platformę do testów algorytmów śledzenia i detekcji.

Dalsze kroki mogą obejmować: \begin{itemize}
    \item stworzenie własnego algorytmu detekcji obiektów
    \item porównanie innych algorytmów śledzenia i detekcji, umożliwiając płynną pracę algorytmu na zaproponowanym sprzęcie
    \item zaprogramowanie śledzenia w dwóch osiach, jednej realizowanej za pośrednictwem gimbala, a drugiej za pomocą drona
    \item stworzenie programu umożliwiającego lądowanie drona na platformach ruchomych, używając np. kodów QR
\end{itemize}

Dokładny wybór ścieżki rozwoju projektu zostanie podjęty wraz z postępem prac oraz dostępnością drona do przeprowadzania testów.

\textit{\textbf{TODO: Stale formatowanie przeglądu wiedzy (imiona i nazwiska)
Przy publikacjach jak jest w jakimś czasopiśmie to ono kursywą, ale nazwę publikacji już nie}}

\section{Detekcja}

Zdecydowałem się na detekcję piłki do piłki nożnej bo jest to prosty obiekt który 
nie będzie wymagał dużego nakładu pracy a nauczę się jak budować dataset i uczyć sieć. 
Wybrałem yolov8 bo jest proste w użyciu i dobrze działa.

yolo z ultralytics -> pc -> 110-140fps, rpi?

onnx działało słabo -> 3-4fps

ncnn + ryzen 5 3600 -> ~22fps 

ncnn + rpi5 bez hat kita ~10-11fps

Zabrałem się za stworzenie gimbala w symulacji, żeby móc symulować śledzenie obiektów 
przez drona przy użyciu gimbala i bez konieczności jeżdżenia na testy


Pomysły na dalsze prace:
\begin{itemize}
    \item śledzenie w symulacji aruco markerów z ruchomym gimbalem
    \item odpalanie detekcji tylko co n klatek
    \item KCF
    \item KCF + Detekcja
\end{itemize}
    
\section{Kinematyka}

Pitch liczony zgodnie z kartezjańskim układem współrzędnych, gdzie oś Z jest skierowana w górę.
Czyli dodatni pitch oznacza, że dron jest pochylony do przodu.

\begin{itemize}
    \item $h$ - wysokość drona nad ziemią
    \item $\alpha_{drone}$ - kąt pitch drona
    \item $\alpha_{gimbal}$ - kąt pitch gimbala
    \item $\gamma$ - yaw drona
    \item $d_{ground}$ - odległość na ziemi od punktu pod dronem do punktu na ziemi, na który patrzy kamera
    \item $r$ - odległość od drona do punktu na ziemi, na który patrzy kamera
    \item $\epsilon$ - pole widzenia kamery w poziomie
    \item $\sigma$ - pole widzenia kamery w pionie
    \item $x$ - szerokość obrazu z kamery w pikselach
\end{itemize}

Dystans po ziemi:

\begin{equation}
    \frac{h}{d_{ground}} = \tan(\alpha_{drone} + \alpha_{gimbal})
\end{equation}

\begin{equation}
    d_{ground} = \frac{h}{\tan(\alpha_{drone} + \alpha_{gimbal})}
\end{equation}

Yaw:

% Wektor od drona do punktu na ziemi w celu skalowania:
% \begin{equation}
%     \sin(\alpha) = \frac{h}{r}
% \end{equation}

% \begin{equation}
%     r = \frac{h}{\sin(\alpha)}
% \end{equation}

Różnica w yaw drona dla odległości poziomej $dx$ obiektu od środka osi symetrii obrazu z kamery:

\begin{equation}
    \frac{dx}{x} = \frac{d \gamma}{\epsilon} % \tan(d \gamma)
\end{equation}

\begin{equation}
    d \gamma = \frac{dx \cdot \epsilon}{x}
\end{equation}

Co dla omawianego przypadku, gdzie pole widzenia kamery w poziomie wynosi $100^o$,
a szerokość obrazu z kamery w pikselach to $640$, daje:

\begin{equation}
    d \gamma = \frac{dx \cdot 100^o}{640}
\end{equation}

Trzeba jakoś zrealizować zatrzymywanie na określonej odległości od obiektu.

Dla cofania (kiedy gimbal osiąga maksymalny kąt pitch w dół):

\begin{equation}
    \frac{d}{h} = \tan (\frac{\sigma}{2})
\end{equation}

\begin{equation}
    d = h \tan (\frac{\sigma}{2})
\end{equation}

A więc

\begin{equation}
    \frac{dy_{px}}{480/2} = \frac{dy_m}{d_m}
\end{equation}

\begin{equation}
    dy_m = \frac{d_m \cdot dy_{px}}{480/2}
\end{equation}

% \chapter{Cel i zakres pracy}
% \input{tekst/cel}

% \chapter{Podsumowanie}
% \label{ch:podsumowanie}
% \input{tekst/podsumowanie}

\glsaddall

    % Bibliografia - musi być
    % Bibliography - must exist
    \bibliografia

    % Strony końcowe - można zakomentować, jeśli zbędne
    % Additional pages - comment out if not needed
    
    % Wykaz symboli i~skrótów - patrz opis w~tekście przykładowym
    \acronymslist
    % Spis rysunków
    \listoffigures
    % Spis tabel
    % \listoftables
    % Załączniki (plik appendices.tex)
    % \easyappendices
\end{document}
%%%%%%%%%%%%%%%%%%%%%%%%%%%%%%%%%%%%%%%%%%%%%%%%%%%%%%%%%%%%%%%%%%%%%%%%%%%